\documentclass[11pt, a4paper]{article}

\usepackage{amsmath}
\usepackage{amsfonts} %Matheschriften
\usepackage{amssymb} %Mathesymbole
%\usepackage{mathptmx} % Einstellung für Schriften und Sonderzeichen in mathematischen Umgebungen
                        % ändert SChriftfont
\usepackage{wasysym} % Stellt diverse Sonderzeichen bereit
\usepackage{siunitx}
\usepackage{float}
\usepackage{microtype}
\usepackage{graphicx}
\usepackage{hyperref}
\usepackage{xcolor}
\usepackage[section]{placeins}
% allows for temporary adjustment of side margins
\usepackage{changepage}
\usepackage{rotating}
\usepackage{tikz}
\usetikzlibrary{arrows,calc,decorations.markings}
\usepackage{numprint}

\usepackage[ngerman]{babel}
\addto\captionsngerman{%
 \renewcommand{\abstractname}{Einleitung}}

\title{Versuch 4: Optik}
\author{Team 4-11: Jascha Fricker, Benedict Brouwer}
\date{17.3.2023}

\pgfkeys{/pgf/number format/.cd, fixed, fixed zerofill, precision=2}

\begin{document}

    \pgfarrowsdeclarecombine{|<}{>|}{latex}{latex}{|}{|}
    \def\Dimline[#1][#2][#3]{
        \begin{scope}[>=latex] % redef arrow for dimension lines
            \draw let \p1=#1, \p2=#2, \n0={veclen(\x2-\x1,\y2-\y1)} in [|<->|,
            decoration={markings, % switch on markings
                    mark=at position 0.5 with {\node[#3] at (0,0) {\DimScale{\n0}};},
            },
            postaction=decorate] #1 -- #2 ;
        \end{scope}
    }

    \def\DimScale#1{\pgfmathparse{round(#1/28.4*10.0)/10.0}\pgfmathresult cm}

    \maketitle

    \tableofcontents

    \newpage

    \section{Einleitung}
    In der Physik spielen Linsen in optischen Versuchsuafbauten eine sehr wichtige Rolle. In diesem Versuch soll es darum gehen, verschiedene 
    Methoden auszuprobieren um den Brechungsindex und Hauptebenenabstand verschiedener Lisnen bzw Linsensysteme zu messen.

    \section{Allgemeine Theorie}
    \FloatBarrier
    In der Physik spielen Linsen in optischen Versuchsuafbauten eine sehr wichtige Rolle. In diesem Versuch soll es darum gehen, verschiedene 
    Methoden auszuprobieren um den Brechungsindex und Hauptebenenabstand verschiedener Lisnen bzw Linsensysteme zu messen.
    \subsection{Autokollimation}
    Wie in Abbildung \ref{fig:autoKollAbb} zu sehen ist, wird die zu untersuchende Linse zwischen Blende und Spiegel gestellt. Nun wird der Abstande $l$ zwischen 
    Linse und Blende so eingestellt, dass auf der Blende welche gleichzeitig als Schirm dient ein Scharfes Bild entsteht. Dieser Vorgang wird wiederholt mit der um $180^\circ$ gedrehten Linse und der Abstand $k$ gemessen.
    Aus diesen Daten lassen Sich Brechungsindex und Hauptebenenabstand wie folgt berechnen:

    \begin{align}
    f' = \frac{k+l-h}{2} \label{eq:autokollBrech}\\
    h = k + l - 2 \cdot f' \label{eq:autokollHaupt}
    \end{align}

    \begin{figure}
        \centering
        \includegraphics[width=0.8\textwidth]{Autokollimation_abb.png}
        \caption{Versuchsaufbau zum Autokollimationsverfahren \cite{OPA}}   % Quelle Fehlt
        \label{fig:autoKollAbb}
    \end{figure}

    \subsection{Besselmethode}
    Bei der Besselmethode wird die zu untersuchende Linse zwischen Blende und Schirm wie in Abbildung \ref{fig:BesselAbb} veranschaulicht gestellt.
    Es gibt nun zwei Positionen, bei denen ein scharrfes Bild auf dem Schirm entsteht. Aus der Differenz dieser Positionen und dem Abstand von Blende und Linse folgt nun für den Brechungsindex und den Hauptebenenabstand:
    \begin{align}
        f' = \frac{1}{4} \cdot [(e-h)-\frac{d^2}{e-h}] \label{eq:besselBrech1}\\
        f = \frac{1}{4} \cdot [\frac{d^2}{e-h}-(e-h)] \label{eq:besselBrech2}
    \end{align}
    

    \begin{figure}
        \centering
        \includegraphics[width=0.8\textwidth]{Bessel_Abb.png}
        \caption{schematischer Versuchsaufbau der Besselmethode \cite{OPA}}   % Quelle Fehlt
        \label{fig:BesselAbb}
    \end{figure}

    \subsection{Dünne Linsen speziefisch}
    Bei Dünnen Linsen kann der Abstand $h = 0$ zwischen den beiden Haupachsen vernachlässigt werden.
    
    \paragraph{Autokollimationsmethode}
    Bei Autokollimationsmethode wird der Aufbau so aufgebaut, dass das Bild und Objekt auf gleicher Position sind. Auf der anderen Seite der Linse reflektiert ein Spiegel die parallelen Strahlen. Wenn das Bild scharf ist, ist die Brennweite
    \begin{align}
        f = \frac{k + l - h}{2}  = d \label{eq:auto}
    \end{align}
    bei dünnen Linsen $h=0$ genau der Abstand $d = k = l$ zwischen Objekt/Bild und Linse.

    \paragraph{Besselmethode}
    Bei der Besselmethode kann die Brennweite
    \begin{align}
        f = \frac{1}{4} \left( e - \frac{d^2}{e} \right) = \frac{1}{4} \left( e - \frac{\left(a_2 - a_1\right)^2}{e} \right) \label{eq:bessel}
    \end{align}
    durch die zwei Positionen der Linse $a_1$ und $a_2$ und den Abstand Objekt-Schirm $e$ bestimmt werden.
    
    \subsection{Optisches System}

    Für die Gesamtbrennweite und Hauptebenen eines Optischen Systems gilt
    \begin{align}
        f &= \frac{f_1' \cdot f_2'}{t - f_1' - f_2'} \label{eq:brenn}\\ 
        \begin{split} \label{eq:ebene}
           z &= \frac{f_1 \cdot t}{t - f_1' - f_2'}  \\
           z' &= \frac{f_2 \cdot t}{t - f_1' - f_2'} \\
           h &= \frac{t}{t - f_1' - f_2'}
        \end{split}
    \end{align}

    \paragraph{Bessel- und Autokollimationsmethode}
    Bei dicken Linsen kann durch zusammenführen der Besselmethode und Autokollimationsmethode die Brennweite $f'$ und der Hauptebenenabstand $h$
    
    \begin{align} \begin{split} \label{eq:dicke}
        f' &= \frac{1}{2} \sqrt{(e-k-l)^2 - d^2} \\
        h &= k + l - \sqrt{(e-k-l)^2 - d^2} \end{split}
    \end{align}
        
    bestimmt werden. Dabei ist $d = a_2 - a_1$ die Distanz zwischen den beiden Positionen der Linse, $e$ der Abstand zwischen Objekt und Schirm, $k$ der Abstand zwischen Linse und Objekt/Bild und $l$ der Abstand zwischen Objekt/Bild bei umgedrehten Linsensystem.

    \section{Messmethode nach Abbe}
    Bei der Messmethode nach Abbe werden die Position der Hauptachsen und Brennweiten der Linsensysteme berechnet. Diese können mit den Gleichungen
    \begin{align}
        \frac{y}{f-a} = \frac{y'}{f} \\
        \frac{y'}{a'-f'} = \frac{-y}{f'} \\
        \beta = \frac{y'}{y} = \frac{f}{a-f} = \frac{f'-a'}{f'}
    \end{align}
    berechnet werden. Durch umstellen der Gleichungen ergeben sich die zwei Beziehungen
    \begin{align}
        g &= f \cdot \left(1 - \frac{1}{\beta}\right) + h_1 \label{eq:dicke1} \\
        g' &= f' \cdot (1 - \beta) + h_2 \label{eq:dicke2} \\
        \text{mit} \ \ \beta &= \frac{y'}{y}
    \end{align}
    Durch fitten an die Daten für den Abstand zum Objekt $g$ bzw zum Bild $g'$ kann die Brennweite $f$ bzw $f'$ und der Hauptebenenabstand $h_1$ und $h_2$ bestimmt werden.

    \section{Versuchdurchführung}

    \subsection{Autokollimationsmethode} % Alles nur ChatGPT

    Die optische Bank wird mit allen notwendigen Komponenten auf Klemmreitern eingerichtet. Als Lichtquelle dient eine Halogenlampe. Ihre Position wird für eine optimale Ausleuchtung justiert. Linse A wird in einem Abstand gleich ihrer Brennweite hinter der Lampe platziert. Das quadratische Raster wird in den Blendenhalter eingefügt und dicht hinter Linse A platziert. Ein weißer Schirm wird in den Blendenhalter eingefügt, sodass eine Hälfte der Blendenöffnung vom Raster und die andere vom Schirm bedeckt ist. Ein Reiter mit der zu untersuchenden Linse und direkt dahinter ein Blendenhalter mit dem Spiegel werden auf die optische Bank gesetzt. Die Positionen werden so lange justiert, bis ein scharfes Bild des Gitters auf dem Schirm abgebildet wird. Dazu muss der Spiegel etwas aus seiner senkrechten Position herausgedreht werden. Dies wird für beide Ausrichtungen des zu untersuchenden Optischen Systems wiederholt.
    Für das Besselverfahren wird der Schirm an die Stelle des Spiegels gestellt und wieder die Abstände so justiert, bis ein scharfes Bild des Gitters auf dem Schirm abgebildet wird. Dies wird für beide Positionen wiederholt.
    Bei der Abbemethode wird mit einem Abstand von etwa 1300 mm zwischen Raster und Schirm begonnen. Das Linsensystem wird so positioniert, dass eine scharfe (vergrößerte) Abbildung des Rasters auf dem Schirm erhalten wird und $g$ und $g'$ bezüglich einer Ablesemarke am Linsensystem gemessen werden. Der zugehörige Abbildungsmaßstab $\beta$ wird bestimmt, indem die Rastergröße des Bildes gemessen wird. Das Linsensystem wird nun zur zweiten Position verschoben, an der ein scharfes (verkleinertes) Bild auf dem Schirm erhalten wird. Auch hier werden $g$ und $g'$ und $\beta$ bestimmt.

    Daraufhin wird der Schirm in Schritten von etwa 25 bis 50 mm in Richtung Lampe verschoben. Das Linsensystem wird jeweils soweit verschoben, bis sich wieder eine scharfe Abbildung ergibt. Dabei werden $g$, $g'$ und $\beta$ für beide möglichen Linsenpositionen bestimmt. Bei diesem Versuchsteil sind erhebliche Unsicherheiten möglich, daher sollte auch die Messunsicherheit zu den Messwerten vermerkt werden.

    Dieser Schritt wird solange wiederholt, bis der Gesamtabstand so klein wird, dass keine scharfe Abbildung mehr möglich ist.


    \section{Ergebnisse}
    \subsection{Aufgabe 1}
    Als erstes wurden alle Linsen angeschaut. Durch beobachten des Millimeterpapiers durch die Linsen konnte zwischen einem vergrößernden Effekt (Konvexe Linse) und einem verkleinerndem Effekt (Konkave Linse) unterschieden werden. Nur die Linse E konnte als Streulise identifiziert werden, alle anderen Linsen waren als Sammellisen zu erkennen.

    \subsection{Einzellinsen}
    Die Ergebnisse der Autokollimationsmethode und der Besselmethode sowie der gewichtete Mittelwert der beiden Methoden wurden mit den Formel (\ref{eq:auto}) und (\ref{eq:bessel}) sind in der Tabelle \ref{tab:ergebnisse} dargestellt.

    \begin{table}[h]
        \centering
        \begin{tabular}{c|c|c}
            Methode & Brennweite B & Brennweite G \\ \hline
            Autokollimation & $10,04(5) \si{\centi\metre}$ & $7,49(7) \si{\centi\metre}$ \\ \hline
            Bessel & $9,97(8) \si{\centi\metre}$ & $7,488(35) \si{\centi\metre}$ \\ \hline
            Mittelwert & $10.02(4) \si{\centi\metre}$ & $7,492(31) \si{\centi\metre}$
        \end{tabular}
        \caption{Brennweiten der Linsen B und G}
        \label{tab:ergebnisse}
    \end{table}
    Der Fehler der Länge wurde mit Tabelle 5 des ABW-Skripts \cite{ABW} berechnet und durch den gewichteten Mittelwert fortgepflanzt. Die Zwischenwerte sind in Tabelle \ref{eq:zwischen} aufgetrage. Die verschiedenen Werte der Methoden stimmen innerhalb der Unsicherheiten für beide Linsen überein. Es ist mit diesen Ergebnissen zu erwarten, dass Linse B eine theoretische Brennweite von $10 \si{\centi\metre}$ hat und Linse G eine Brennweite von $7,5 \si{\centi\metre}$ haben sollte, was auch mit den Unsicherheiten übereinstimmt.

    \begin{table}[h]
        \centering
        \begin{tabular}{c|c|c}
            Methode & Brennweite B & Brennweite G \\ \hline
            Autokollimation l & $10,06(4) \si{\centi\metre}$ & $7,44(4) \si{\centi\metre}$ \\ \hline
            Autokollimation k & $9,91(11) \si{\centi\metre}$ & $7,57(4) \si{\centi\metre}$ \\ \hline
            Bessel Abstand $e$ & $41,40(10) \si{\centi\metre}$ & $41,40(10) \si{\centi\metre}$ \\ \hline
            Abstand $d$ & $38,29(22) \si{\centi\metre}$ & $21,77(5) \si{\centi\metre}$
        \end{tabular}
        \caption{Zwischenwerte }
        \label{tab:zwischen}
    \end{table}

    \subsection{Linsensystem}
    Wir haben das Linsensystem E-G mit $30$ mm Abstand für unsere Messungen benutzt. Wobei die Linse E die erste im Strahlengang ist. Dies ist genau anders herum wie in der Aufgabenstellung.

    \paragraph{Autokollimation- und Besselmethode}
    Mit Gleichung \ref{eq:dicke} konnte aus den Messwerten der beiden Methoden die Brennweite und der Hauptebenenabstand
    \begin{align}
        f' &= 13,10(20) \si{\centi\metre} \\
        h &= 0,3(6) \si{\centi\metre}
    \end{align}
    bestimmt werden. Der Fehler der Länge wurde mit Tabelle 5 des ABW-Skripts \cite{ABW} berechnet und der gewichtete Mittelwert wurde genutzt. Als zwischenergebnisse wurden $e = 64,80(12) \si{\centi\metre}$, $k = 6,80(12) \si{\centi\metre}$, $l = 19,08(16) \si{\centi\metre}$ und $d = 28,78(18) \si{\centi\metre}$ berechnet.

    \paragraph{Messmethode nach Abbe}
    Um die Brennweiten und Hauptebenenabstände zu bestimmen,
    \begin{align}
        f &= 14,22(34) \si{\centi\metre} \\
        f' &= -12,74(19) \si{\centi\metre} \\
        h_1 &= 6,81(73) \si{\centi\metre} \\
        h_2 &= 6,82(87) \si{\centi\metre}
    \end{align}
    
    wurden die Messwerte für den Abstand zum Objekt $g$ und zum Bild $g'$ mit den Formeln (\ref{eq:dicke1}) und (\ref{eq:dicke2}) gefittet. Die Fits sind in \ref{fig:fit1} und \ref{fig:fit2} dargestellt. Alle Angaben sind relativ zur Linse E. Der gewichtete Mittelwert der Beträge der zwei Brennweiten ist $f = 13,29(70) \si{\centi\metre}$. Dieser Mittelwert stimmt sehr gut innerhalb der Unsicherheit mit dem Wert der Autokollimation und der Besselmethode überein. Bei den Hauptebenen gibt es hingegen bei der Abbemethode sehr große Unsicherheiten. Dies kommt wahrscheinlich von der großen Unsicherheit der Größenmessung des Bildes. Dass der Abstand h innerhalb der Unsicherheiten von $h_1$ und $h_2$ liegt, sagt nicht viel aus.

    \begin{figure}[h]
        \centering
        \includegraphics[width=0.8\textwidth]{g.pdf}
        \caption{Fit der Messwerte für den Abstand zum Objekt}
        \label{fig:fit1}
    \end{figure}

    \begin{figure}[h]
        \centering
        \includegraphics[width=0.8\textwidth]{g_prime.pdf}
        \caption{Fit der Messwerte für den Abstand zum Bild}
        \label{fig:fit2}
    \end{figure}

    % Bild des Aufbaus


    \begin{figure}[h]
        \centering
        \begin{tikzpicture}[scale=0.3]
            % Define lens properties
            \def\lensRadius{10}
            \def\lensHeight{12}
            \def\lensWidth{1}
            \def\focalLengthOne{-5} % concave lens
            \def\focalLengthTwo{5} % convex lens
            \def\startAngle{asin(\lensHeight/\lensRadius)}
            \pgfmathsetmacro{\lensRadius}{50}
            \pgfmathsetmacro{\lensHeight}{10}
            \pgfmathsetmacro{\startAngle}{asin(\lensHeight/\lensRadius)}


            % Draw lenses
            \draw [fill=blue!15]  (3,\lensHeight)
            arc[start angle=180-\startAngle,delta angle=2*\startAngle,radius=\lensRadius]
            arc[start angle=-\startAngle,delta angle=2*\startAngle,radius=\lensRadius]
             -- cycle; % to get a better line end
            \draw [fill=blue!15] (\lensWidth,\lensHeight)
            arc[start angle=180-\startAngle,delta angle=2*\startAngle,radius=\lensRadius] -- (-\lensWidth, -\lensHeight)
            arc[start angle=-\startAngle,delta angle=2*\startAngle,radius=\lensRadius]
             -- cycle; % to get a better line end
            % Draw principal planes
            \draw [dashed] (6.806, -1*\lensHeight) -- (6.806, 1*\lensHeight);
            \draw [dashed] (6.827, -1*\lensHeight) -- (6.827, 1*\lensHeight);

            % Draw focal points
            \filldraw (- 14.217 + 6.806,0) circle (4pt);
            \filldraw (12.741 + 6.827 ,0) circle (4pt);

            % Label lenses
            \node at (0,-14) (E) {E};
            \node at (3,-14) (G) {G};

            % Label principal planes and focal points
            \node at (6.806,-14) (P1) {};
            \node at (6.806,-13) {$H_1$};
            \node at (- 14.217 + 6.806,-14) (F1) {$F_1$};
            \node at (6.827,-14) (P2) {$H_2$};
            \node at (12.741 + 6.827 ,-14) (F2) {$F_2$};

            \Dimline[($(E)+(0,-2)$)][($(G)+(0,-2)$)][below];
            \Dimline[($(P2)+(0, 3)$)][($(F2)+(0, 3)$)][above];
            \Dimline[($(G)+(0,-2)$)][($(P2)+(0,-2)$)][above];
            \Dimline[($(P1)+(0, 2)$)][($(F1)+(0, 2)$)][above];
            \Dimline[($(E)+(0,-4)$)][($(P1)+(0,-4)$)][below];

            % Draw optical axis
            \draw [->](-10,0)--(25,0);

            % Label optical axis
            %\node at(30,-3){Optical Axis};

        \end{tikzpicture}
        \caption{Linsensystem E-G 30mm}
        \label{fig:aufbau}
    \end{figure}


    \subsubsection{Berechnung der Brennweiten und Hauptebenen}
    \paragraph{Brennweite}
    Durch umstellen der Gleichung (\ref{eq:brenn}) nach $f_E$ kann die Brennweite der ersten Linse E bei durch Aufgaben 2 und 3 gegebener Brennweite $f_G = 7.492(31) \si{\centi\metre}$ und $f' = - 13,10(20) \si{\centi\metre}$ berechnet werden. Der Abstand $t = 3 \si{\centi\metre}$ ist auch gegeben.

    \begin{align}
        f_E = \frac{f' \left(t - f'\right)}{f' - f_G} = 10.24(20) \si{\centi\metre}
    \end{align}

    Diese Werte für $f_E$ stimmt nicht ganz innerhalb der Unsicherheit mit dem Theoriewert von $f_E = 10 \si{\centi\metre}$ überein. Trotzdem ist die Abbweichugn nicht sehr groß. Wahrsceinlich ist die berechnete Unsicherheit zu klein, da es beim Messen der Daten große Bereiche gab, in denen das Bild relativ scharf war, der genau Punkt also nur schwer bestimmt werden konnte. Dieser Fehler wurde durch die Unsicherheit nicht berücksichtigt.

    \paragraph{Hauptebenen}
    Die Abstand der Haubtebenen von den jeweiligen Linsen $z$ und $z'$ sowie der relative Abstand $h$ kann durch Gleichungen (\ref{eq:ebene}) berechnet werden. Dabei gilt $f_1' = -f_1 = f_E' = -10 \si{\centi\metre}$ und $f_2' = f_G' = 7,492(31) \si{\centi\metre}$.
    \begin{align}
        z &= \frac{f_1 \cdot t}{t - f_1' - f_2'} = 5,447(89) \si{\centi\metre} \\
        z' &= \frac{f_2 \cdot t}{t - f_1' - f_2'} = 4,081(74) \si{\centi\metre} \\
        h &= \frac{t}{t - f_1' - f_2'} = 0,5447(89) \si{\centi\metre}
    \end{align}
    Der Wert von $h$ rechnerisch und graphisch liegen innerhalb der Unsicherheit, die Werte von $z$ und $z'$ sind jedoch nicht innerhalb der Unsicherheiten von $h_1$ und $h_2 - 3 \si{\centi\metre}$. Dies kann an einer Ungenauen Bestimmung durch die Abbe Methode liegen, wie bei der Berechnung der Brennweite beschrieben.

    \paragraph{Simulation}
    In Abbildung \ref{fig:sim} ist die Simulation des Strahlenverlaufs für die Optik G-E veranschaulicht. Da unser Aufbau genau andersherum ist, muss die Optik G-E umgedreht werden. Die Simulation zeigt, dass die Lage der entscheidenden Punkte qualitativ mit den von uns bestimmten Werten übereinstimmen. Die Hauptebenen sind in der Simulation etwas weiter auseinander als bei uns. Dies kann an der Ungenauigkeit der Abbe Methode liegen.

    \begin{figure}[h]
        \centering
        \includegraphics[width=0.8\textwidth]{G-E-30.pdf}
        \caption{Simulation des Strahlenverlaufs für die Optik G-E}
        \label{fig:sim}
    \end{figure}

    \section{Diskussion}

    Insgesamt war die Bestimmung der Eigenschaften verschiederner Linsen und Linsensysteme trotz der relativ ungenauen Bestimmung der Punkte mit scharfer Abbildung relativ erfolgreich. Die meisten Messwerte passen innerhalb der Unsicherheiten zusammen, nur die berechneten Werte für die Brennweite der ersten Linse E und die Abstände der Hauptebenen von den Linsen sind nicht ganz innerhalb der Unsicherheiten. Die Autokollimations- und Besselmethode haben sich als sehr nützlich erwiesen, da sie relativ einfach zu handhaben sind und die Messwerte relativ genau sind. Die Abbe Methode ist hingegen ungenauer, aber nur hier kann die genau Position der Hauptebenen und der zwei verschiedenen Brennweiten bestimmt werden.

    \section{Fragen}
    \begin{itemize}
        \item Beim Bessel-Verfahren gibt es zwei Positionen des Linsensystems, an denen eine scharfe Abbildung möglich ist. Aufgrund der Umkehrbarkeit des Strahlengangs entsprechen die objektseitigen Größen im ersten Fall gerade den bildseitigen Größen im Zweiten. Das heißt die Abbildungsmaßstäbe sind genau die Kehrwerte voneinander.
        \item In einem Projektionsapparat wird die Lampe hinter einem Hohlspiegel angeordnet. Der Kondensor ist eine Linsenkombination, die wie eine Sammellinse wirkt und so angeordnet ist, dass das Dia gut ausgeleuchtet ist und das gesamte vom Dia ausgehende Licht ins Objektiv trifft. Auf diese Weise entsteht ein vollständiges, höhen- und seitenverkehrtes, vergrößertes Bild vom Original auf der Leinwand. Der Schemtaische Aufbau wird in Abbildung \ref{fig:projektionsapparat} dargestellt.
        \item Wenn sich das Objekt in einem Abstand von a = 0,5 · f von der objektseitigen Hauptebene einer Sammellinse befindet, dann ist a < f. In diesem Fall entsteht ein virtuelles Bild, das nicht auf dem Kopf stehen und größer als der Gegenstand ist. Der Abbildungsmaßstab ist 2 und die Position $a'= f$ ist genau der gegenstandsseitige Brennpunkt.
        \item Die Gesamtbrennweite eines Systems zweier Sammellinsen gleicher Brennweite kann durch die Verwendung der Linsengleichung berechnet werden. Die Linsengleichung lautet: $\frac{1}{f} = \frac{1}{f_1} + \frac{1}{f_2} - \frac{t}{f_1 f_2}$, wobei $f$ die Gesamtbrennweite des Systems ist, $f_1$ und $f_2$ die Brennweiten der beiden Linsen und $t$ der Abstand zwischen den beiden Linsen.
        In diesem Fall haben beide Linsen die gleiche Brennweite ($f_1 = f_2$), so dass sich die Gleichung vereinfacht zu: $f = \frac{f_1^2}{t-2f_1}$. 
    \end{itemize}

    \begin{figure}[h]
        \centering
        \includegraphics[width=0.8\textwidth]{Condensor-1-de.svg.png}
        \caption{Schematischer Aufbau eines Projektionsapparates nach \cite{projektionsapparat}}
        \label{fig:projektionsapparat}
    \end{figure}





    \bibliographystyle{plain}
    \bibliography{literature}

\end{document}