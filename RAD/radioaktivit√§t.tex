\documentclass[11pt, a4paper]{article}

\usepackage{amsmath}
\usepackage{amsfonts} %Matheschriften
\usepackage{amssymb} %Mathesymbole
%\usepackage{mathptmx} % Einstellung für Schriften und Sonderzeichen in mathematischen Umgebungen
                        % ändert SChriftfont
\usepackage{wasysym} % Stellt diverse Sonderzeichen bereit
\usepackage{siunitx}
\usepackage{float}
\usepackage{microtype}
\usepackage{graphicx}
\usepackage{hyperref}
\usepackage{xcolor}
\usepackage[section]{placeins}
% allows for temporary adjustment of side margins
\usepackage{changepage}
\usepackage{rotating}


\usepackage[ngerman]{babel}
\addto\captionsngerman{%
 \renewcommand{\abstractname}{Einleitung}}

\title{Versuch 3: Radioaktivität}
\author{Team 4-11: Jascha Fricker, Benedict Brouwer}

\begin{document}
    \maketitle

    \tableofcontents

    \newpage

    \section{Theorie}

    \section{Ergebnisse}

    \section{Diskussion}
    \sectio{Fragen}
    \begin{enumerate}
        \item Ein Elektronvolt ist die Menge an kinetischer Energie, die ein einzelnes Elektron gewinnt oder verliert, wenn es sich von der Ruhe aus durch einen elektrischen Potentialunterschied von einem Volt im Vakuum beschleunigt.
        Ein kosmisches Teilchen mit einer Energie von $4 \cdot 10^{12}$ GeV hat eine Energie von $4 \cdot 10^{12} \cdot 10^9$ eV oder $4 \cdot 10^{21}$ eV. Da 1 eV = $1.602 \times 10^{-19}$ J ist3, hat das kosmische Teilchen eine Energie von $4 \cdot 10^{21} \cdot 1.602 \times 10^{-19}$ J oder etwa $6.408 \times 10^2$ J.        
        Zum Vergleich: Eine AA-Batterie enthält etwa 4 Wattstunden oder etwa 14.400 Joule an Energie.
        \item Die Peaks in einem Experiment haben eine Breite aufgrund von verschiedenen Faktoren wie der Auflösung des Detektors, der statistischen Natur der Messungen und der Breite der Energieverteilung der emittierten Strahlung.
        Nicht alle γ-Quanten einer bestimmten Energie werden in einem einzigen Kanal gezählt, da die Energieauflösung des Detektors nicht unendlich ist und es immer eine gewisse Unschärfe in der Messung gibt.
        Die Halbwertsbreite (engl. Full Width at Half Maximum, FWHM) ist ein Maß für die Breite eines Peaks in einem Spektrum. Es wird definiert als die Breite des Peaks bei der halben Höhe des Maximums.
        \item Das Na^{22}-ZerfallsSpektrum besteht aus zwei Hauptpeaks. Der erste Peak bei 511 keV entsteht durch die Annihilation von Elektronen und Positronen. Der zweite Peak bei 1.27 MeV entsteht durch einen Kernenergieübergang.
        \item 
    \end{enumerate}

    \bibliographystyle{plain}
    \bibliography{literature}

\end{document}